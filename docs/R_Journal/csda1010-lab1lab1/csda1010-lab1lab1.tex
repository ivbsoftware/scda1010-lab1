% !TeX root = RJwrapper.tex
\title{Ranking Applications for Nursery Schools}
\author{by Viviane Adohouannon, Kate Alexander, Juan Arangote, Dian Azbel, Igor Baranov}

\maketitle

\abstract{%
Nursery Database was derived from a hierarchical decision model
originally developed to rank applications for nursery schools.Out task
was to develop a classification model to develop a reliable
recomendation algorithm to predict is a specific student is sutable to
be addmitted to a nursing school.
}

% Any extra LaTeX you need in the preamble

\hypertarget{introduction}{%
\section{Introduction}\label{introduction}}

This section may contain a figure such as Figure \ref{figure:rlogo}.

\begin{figure}[htbp]
  \centering
  \includegraphics{Rlogo}
  \caption{The logo of R.}
  \label{figure:rlogo}
\end{figure}

\hypertarget{background}{%
\subsection{Background}\label{background}}

Introductory section which may include references in parentheses
\citep{R}, or cite a reference such as \citet{R} in the text.

Nursery Data is described as \ldots{}Sed ut dui dui. Vestibulum vel
velit at mauris auctor gravida condimentum porttitor metus. Integer
tempus nunc ac sem pharetra volutpat. Fusce vitae eleifend leo. Ut vel
tempor nibh. Proin eget fermentum leo. Mauris pharetra vitae sem eget
dictum. Sed at neque vitae metus lobortis luctus. Morbi sapien diam,
vulputate sed diam vitae, pharetra accumsan mauris. Curabitur pretium
nulla turpis, porta porta odio tempor vitae. In vehicula volutpat dui et
consequat. Nam id lorem molestie sapien ultrices elementum. In vehicula
metus elit, nec rhoncus est efficitur et. Proin ex tellus, vestibulum a
eros at, maximus euismod justo.

Source: \url{http://archive.ics.uci.edu/ml/datasets/Nursery}

\begin{verbatim}
| class values

not_recom, recommend, very_recom, priority, spec_prior

| attributes

parents:     usual, pretentious, great_pret.
has_nurs:    proper, less_proper, improper, critical, very_crit.
form:        complete, completed, incomplete, foster.
children:    1, 2, 3, more.
housing:     convenient, less_conv, critical.
finance:     convenient, inconv.
social:      nonprob, slightly_prob, problematic.
health:      recommended, priority, not_recom.
\end{verbatim}

\hypertarget{objective-and-hypothesis}{%
\subsection{Objective and Hypothesis}\label{objective-and-hypothesis}}

Lorem ipsum dolor sit amet, consectetur adipiscing elit, sed do eiusmod
tempor incididunt ut labore et dolore magna aliqua. Ut enim ad minim
veniam, quis nostrud exercitation ullamco laboris nisi ut aliquip ex ea
commodo consequat. Duis aute irure dolor in reprehenderit in voluptate
velit esse cillum dolore eu fugiat nulla pariatur. Excepteur sint
occaecat cupidatat non proident, sunt in culpa qui officia deserunt
mollit anim id est laborum.

\hypertarget{assessment-of-situation}{%
\subsection{Assessment of Situation}\label{assessment-of-situation}}

At vero eos et accusamus et iusto odio dignissimos ducimus qui
blanditiis praesentium voluptatum deleniti atque corrupti quos dolores
et quas molestias excepturi sint occaecati cupiditate non provident,
similique sunt in culpa qui officia deserunt mollitia animi, id est
laborum et dolorum fuga. Et harum quidem rerum facilis est et expedita
distinctio. Nam libero tempore, cum soluta nobis est eligendi optio
cumque nihil impedit quo minus id quod maxime placeat facere possimus,
omnis voluptas assumenda est, omnis dolor repellendus. Temporibus autem
quibusdam et aut officiis debitis aut rerum necessitatibus saepe eveniet
ut et voluptates repudiandae sint et molestiae non recusandae.

\hypertarget{plan}{%
\subsection{Plan}\label{plan}}

Et harum quidem rerum facilis est et expedita distinctio. Nam libero
tempore, cum soluta nobis est eligendi optio cumque nihil impedit quo
minus id quod:

\begin{itemize}
\tightlist
\item
  Cras sit amet lacus luctus massa lobortis sagittis.
\item
  Curabitur convallis nisi non fringilla mattis.
\item
  Etiam auctor massa nec orci hendrerit convallis.
\item
  Nunc et tortor eu nisl gravida tempor a a lorem.
\item
  Morbi sit amet sem posuere, aliquam ex ut, lacinia nulla.
\end{itemize}

\hypertarget{data-understanding}{%
\section{Data understanding}\label{data-understanding}}

Harum quidem rerum facilis est et expedita distinctio. Nam libero
tempore, cum soluta nobis est eligendi optio cumque nihil impedit quo
minus id quod.

\hypertarget{preparation}{%
\subsection{Preparation}\label{preparation}}

\begin{table}[ht]
\centering
\scalebox{0.75}{
\begin{tabular}{rrlllllllll}
  \hline
 & X & parents & has\_nurs & form & children & housing & finance & social & health & class \\ 
  \hline
1 &   1 & usual & proper & complete & 1 & convenient & convenient & nonprob & recommended & recommend \\ 
  2 &   2 & usual & proper & complete & 1 & convenient & convenient & nonprob & priority & priority \\ 
  3 &   3 & usual & proper & complete & 1 & convenient & convenient & nonprob & not\_recom & not\_recom \\ 
  4 &   4 & usual & proper & complete & 1 & convenient & convenient & slightly\_prob & recommended & recommend \\ 
  5 &   5 & usual & proper & complete & 1 & convenient & convenient & slightly\_prob & priority & priority \\ 
  6 &   6 & usual & proper & complete & 1 & convenient & convenient & slightly\_prob & not\_recom & not\_recom \\ 
  7 &   7 & usual & proper & complete & 1 & convenient & convenient & problematic & recommended & priority \\ 
  8 &   8 & usual & proper & complete & 1 & convenient & convenient & problematic & priority & priority \\ 
  9 &   9 & usual & proper & complete & 1 & convenient & convenient & problematic & not\_recom & not\_recom \\ 
  10 &  10 & usual & proper & complete & 1 & convenient & inconv & nonprob & recommended & very\_recom \\ 
   \hline
\end{tabular}
}
\caption{\tt Nursery Data Dataset (head)} 
\end{table}

Summary of Nursery Data set is extracted by the folloving `summary'
function, the results are presented in Table \ref{table:dsum1} and Table
\ref{table:dsum2}.

We can conclude that: consectetur adipiscing elit. Aenean magna urna,
sodales vel blandit sed, condimentum sit amet ante. Phasellus pulvinar
ullamcorper porttitor. Cras vitae ipsum in magna condimentum malesuada
ut at massa. Duis quis quam faucibus, euismod lacus sit amet,
scelerisque odio.

\begin{table}[ht]
\centering
\begin{tabular}{rlllll}
  \hline
 &       X &        parents &        has\_nurs &         form & children \\ 
  \hline
X & Min.   :    1   & great\_pret :4320   & critical   :2592   & complete  :3240   & 1   :3240   \\ 
  X.1 & 1st Qu.: 3241   & pretentious:4320   & improper   :2592   & completed :3240   & 2   :3240   \\ 
  X.2 & Median : 6480   & usual      :4320   & less\_proper:2592   & foster    :3240   & 3   :3240   \\ 
  X.3 & Mean   : 6480   &  & proper     :2592   & incomplete:3240   & more:3240   \\ 
  X.4 & 3rd Qu.: 9720   &  & very\_crit  :2592   &  &  \\ 
  X.5 & Max.   :12960   &  &  &  &  \\ 
   \hline
\end{tabular}
\caption{\tt Summary of Nursery Data Dataset, columns 1-5} 
\label{table:dsum1}
\end{table}
\begin{table}[ht]
\centering
\begin{tabular}{rllll}
  \hline
 &       housing &       finance &           social &         health \\ 
  \hline
X & convenient:4320   & convenient:6480   & nonprob      :4320   & not\_recom  :4320   \\ 
  X.1 & critical  :4320   & inconv    :6480   & problematic  :4320   & priority   :4320   \\ 
  X.2 & less\_conv :4320   &  & slightly\_prob:4320   & recommended:4320   \\ 
  X.3 &  &  &  &  \\ 
  X.4 &  &  &  &  \\ 
  X.5 &  &  &  &  \\ 
   \hline
\end{tabular}
\caption{\tt Summary of Nursery Data Dataset, columns 6-9} 
\label{table:dsum2}
\end{table}

Distribution of Class attribute presented in Figure \ref{fig:classd}.
Morbi volutpat augue vitae porta lobortis. Integer sit amet neque vel
risus aliquam scelerisque et eget est. Cras maximus ex nec pharetra
dictum. Vivamus vehicula ante sodales massa rhoncus, et blandit tortor
interdum. Pellentesque aliquam ligula eu justo porttitor, non fringilla
erat vehicula. Pellentesque et dolor non nunc aliquet euismod. Vivamus
vel malesuada lorem. Fusce eget mauris a nulla sollicitudin eleifend eu
auctor ligula. Sed nec dictum lorem. Vestibulum bibendum ultrices lorem,
id fermentum felis tincidunt eu. Curabitur ipsum justo, dictum id
finibus vitae, pulvinar non tortor. Curabitur vel mi a urna gravida
commodo vitae vel libero. Maecenas imperdiet sed diam eget viverra.

\begin{Schunk}
\begin{Sinput}
prop.table((table(nursery_data$class)))
\end{Sinput}
\begin{Soutput}
#> 
#>   not_recom    priority   recommend  spec_prior  very_recom 
#> 0.333333333 0.329166667 0.000154321 0.312037037 0.025308642
\end{Soutput}
\begin{Sinput}
ggplot(nursery_data, aes(x=as.factor(class) )) + 
  geom_bar(aes(y= (..count..)/sum(..count..)),color="blue",fill=rgb(0.2,0,0.5)) +
  theme(legend.position = "none") +
  scale_y_continuous(labels=scales:::percent) +
  labs(x = "class",y="total")
\end{Sinput}
\begin{figure}

{\centering \includegraphics{csda1010-lab1lab1_files/figure-latex/classd-1} 

}

\caption[Distribution of Class attribute]{Distribution of Class attribute}\label{fig:classd}
\end{figure}
\end{Schunk}

The following code (Figure \ref{fig:classdp}) renders distribution of
class attribute depending on parents attribute. Pellentesque aliquam
ligula eu justo porttitor, non fringilla erat vehicula. Pellentesque et
dolor non nunc aliquet euismod. Vivamus vel malesuada lorem. Fusce eget
mauris a nulla sollicitudin eleifend eu auctor ligula. Sed nec dictum
lorem. Vestibulum bibendum ultrices lorem, id fermentum felis tincidunt
eu. Curabitur ipsum justo, dictum id finibus vitae, pulvinar non tortor.
Curabitur vel mi a urna gravida commodo vitae vel libero. Maecenas
imperdiet sed diam eget viverra.

\begin{Schunk}
\begin{Sinput}
ggplot(nursery_data,aes(x=factor(parents),fill=factor(class)))+
  geom_bar(position="dodge")
\end{Sinput}
\begin{figure}

{\centering \includegraphics{csda1010-lab1lab1_files/figure-latex/classdp-1} 

}

\caption[Distribution of Class attribute depending on parents]{Distribution of Class attribute depending on parents}\label{fig:classdp}
\end{figure}
\end{Schunk}

The following code (Figure \ref{fig:classdh}) renders distribution of
class attribute depending on `health' attribute. Pellentesque aliquam
ligula eu justo porttitor, non fringilla erat vehicula. Pellentesque et
dolor non nunc aliquet euismod. Vivamus vel malesuada lorem. Fusce eget
mauris a nulla sollicitudin eleifend eu auctor ligula. Sed nec dictum
lorem. Vestibulum bibendum ultrices lorem, id fermentum felis tincidunt
eu. Curabitur ipsum justo, dictum id finibus vitae, pulvinar non tortor.
Curabitur vel mi a urna gravida commodo vitae vel libero. Maecenas
imperdiet sed diam eget viverra.

\begin{Schunk}
\begin{Sinput}
nursery_data$health <- as.factor(nursery_data$health)
ggplot(data = nursery_data, mapping = aes(x = class, fill = health)) +
  geom_bar()
\end{Sinput}
\begin{figure}

{\centering \includegraphics{csda1010-lab1lab1_files/figure-latex/classdh-1} 

}

\caption[Distribution of Class attribute depending on health]{Distribution of Class attribute depending on health}\label{fig:classdh}
\end{figure}
\end{Schunk}

The following code (Figure \ref{fig:classdi}) renders distribution of
class attribute depending on `'social' attribute. Pellentesque aliquam
ligula eu justo porttitor, non fringilla erat vehicula. Pellentesque et
dolor non nunc aliquet euismod. Vivamus vel malesuada lorem. Fusce eget
mauris a nulla sollicitudin eleifend eu auctor ligula. Sed nec dictum
lorem. Vestibulum bibendum ultrices lorem, id fermentum felis tincidunt
eu. Curabitur ipsum justo, dictum id finibus vitae, pulvinar non tortor.
Curabitur vel mi a urna gravida commodo vitae vel libero. Maecenas
imperdiet sed diam eget viverra.

\begin{Schunk}
\begin{Sinput}
ggplot(nursery_data, aes(social, fill=class)) + 
  geom_bar(aes(y = (..count..)/sum(..count..)), alpha=0.9) +
  facet_wrap(~parents) + 
  scale_fill_brewer(palette = "Dark2", direction = -1) +
  scale_y_continuous(labels=scales:::percent, breaks=seq(0,0.4,0.05)) +
  ylab("Percentage") +
  theme_bw() +
  theme(plot.title = element_text(hjust = 0.5)) +
  theme(axis.text.x = element_text(angle = 90, hjust = 1))
\end{Sinput}
\begin{figure}

{\centering \includegraphics{csda1010-lab1lab1_files/figure-latex/classdi-1} 

}

\caption[Distribution of Class attribute depending on income]{Distribution of Class attribute depending on income}\label{fig:classdi}
\end{figure}
\end{Schunk}

The following code (Figure \ref{fig:classdi}) renders distribution of
`class' attribute depending on `nursery' attribute. Pellentesque aliquam
ligula eu justo porttitor, non fringilla erat vehicula. Pellentesque et
dolor non nunc aliquet euismod. Vivamus vel malesuada lorem. Fusce eget
mauris a nulla sollicitudin eleifend eu auctor ligula. Sed nec dictum
lorem. Vestibulum bibendum ultrices lorem, id fermentum felis tincidunt
eu. Curabitur ipsum justo, dictum id finibus vitae, pulvinar non tortor.
Curabitur vel mi a urna gravida commodo vitae vel libero. Maecenas
imperdiet sed diam eget viverra.

\begin{Schunk}
\begin{Sinput}
ggplot(nursery_data, aes(form, fill=class)) + 
  geom_bar(aes(y = (..count..)/sum(..count..)), alpha=0.9) +
  facet_wrap(~has_nurs) + 
  scale_fill_brewer(palette = "Dark2", direction = -1) +
  scale_y_continuous(labels=scales:::percent, breaks=seq(0,0.4,0.05)) +
  ylab("Percentage") +
  theme_bw() +
  theme(plot.title = element_text(hjust = 0.5)) +
  theme(axis.text.x = element_text(angle = 90, hjust = 1))
\end{Sinput}
\begin{figure}

{\centering \includegraphics{csda1010-lab1lab1_files/figure-latex/classdn-1} 

}

\caption[Distribution of Class attribute depending on nursery]{Distribution of Class attribute depending on nursery}\label{fig:classdn}
\end{figure}
\end{Schunk}

The conclusion is: aliquam ligula eu justo porttitor, non fringilla erat
vehicula. Pellentesque et dolor non nunc aliquet euismod. Vivamus vel
malesuada lorem. Fusce eget mauris a nulla sollicitudin eleifend eu
auctor ligula. Sed nec dictum lorem. Vestibulum bibendum ultrices lorem,
id fermentum felis tincidunt eu. Curabitur ipsum justo, dictum id
finibus vitae, pulvinar non tortor. Curabitur vel mi a urna gravida
commodo vitae vel libero. Maecenas imperdiet sed diam eget viverra.

\hypertarget{modeling}{%
\section{Modeling}\label{modeling}}

\hypertarget{splitting-the-dataset-into-train-and-test}{%
\subsection{Splitting the dataset into train and
test}\label{splitting-the-dataset-into-train-and-test}}

We are splitting the dataset in such a way, that train and test sets
would have similar distribution of the `class' attribute.

\begin{Schunk}
\begin{Sinput}
train.rows<- createDataPartition(y= nursery_data$class, p=0.9, list = FALSE)
train.data<- nursery_data[train.rows,]
prop.table((table(train.data$class)))
\end{Sinput}
\begin{Soutput}
#> 
#>    not_recom     priority    recommend   spec_prior   very_recom 
#> 0.3332761872 0.3291616664 0.0001714384 0.3120178296 0.0253728785
\end{Soutput}
\end{Schunk}

\begin{Schunk}
\begin{Sinput}
test.data<- nursery_data[-train.rows,]
prop.table((table(test.data$class)))
\end{Sinput}
\begin{Soutput}
#> 
#>  not_recom   priority  recommend spec_prior very_recom 
#> 0.33384853 0.32921175 0.00000000 0.31221020 0.02472952
\end{Soutput}
\end{Schunk}

The following code (Figure \ref{fig:testd}) renders distribution of
`class' attribute depending on `nursery' attribute. Pellentesque aliquam
ligula eu justo porttitor, non fringilla erat vehicula. Pellentesque et
dolor non nunc aliquet euismod. Vivamus vel malesuada lorem. Fusce eget
mauris a nulla sollicitudin eleifend eu auctor ligula. Sed nec dictum
lorem. Vestibulum bibendum ultrices lorem, id fermentum felis tincidunt
eu. Curabitur ipsum justo, dictum id finibus vitae, pulvinar non tortor.
Curabitur vel mi a urna gravida commodo vitae vel libero. Maecenas
imperdiet sed diam eget viverra.

\begin{Schunk}
\begin{Sinput}
ggplot(test.data, aes(x=as.factor(class))) + 
  geom_bar(aes(y = (..count..)/sum(..count..)),width=0.4,
  color="red", fill=rgb(0.9,1,0.7) )+theme(legend.position = "none") + 
  labs(x = "class",y="total")+scale_y_continuous(labels=scales:::percent)
\end{Sinput}
\begin{figure}

{\centering \includegraphics{csda1010-lab1lab1_files/figure-latex/testd-1} 

}

\caption[Distribution of Class attribute depending on nursery]{Distribution of Class attribute depending on nursery}\label{fig:testd}
\end{figure}
\end{Schunk}

\hypertarget{decision-tree-model-fit}{%
\subsection{Decision Tree model fit}\label{decision-tree-model-fit}}

As a first step we will train a Desigion Tree model. This model is known
to be computationally fast, but not very precise. We will use all
default parameters and all attributes of thetrain dataset. Resuling tree
is presented in Figure \ref{fig:dtree}. It shows that Fusce eget mauris
a nulla sollicitudin eleifend eu auctor ligula. Sed nec dictum lorem.
Vestibulum bibendum ultrices lorem, id fermentum felis tincidunt eu.
Curabitur ipsum justo, dictum id finibus vitae, pulvinar non tortor.
Curabitur vel mi a urna gravida commodo vitae vel libero. Maecenas
imperdiet sed diam eget viverra.

\begin{Schunk}
\begin{Sinput}
fitdt <- rpart(as.factor(class)~., method="class", data=train.data)
fancyRpartPlot(fitdt, main = "", sub = "")
\end{Sinput}
\begin{figure}

{\centering \includegraphics{csda1010-lab1lab1_files/figure-latex/dtree-1} 

}

\caption[Decision tree diagram]{Decision tree diagram}\label{fig:dtree}
\end{figure}
\end{Schunk}

\hypertarget{decision-tree-model-evaluation}{%
\subsection{Decision Tree model
evaluation}\label{decision-tree-model-evaluation}}

\begin{Schunk}
\begin{Sinput}
dtPrediction <- predict(fitdt, test.data, type = "class")
\end{Sinput}
\end{Schunk}

\begin{Schunk}
\begin{Sinput}
confMat <- table(dtPrediction,test.data$class)
confMat
\end{Sinput}
\begin{Soutput}
#>             
#> dtPrediction not_recom priority recommend spec_prior very_recom
#>   not_recom        432        0         0          0          0
#>   priority           0      325         0         11         32
#>   recommend          0        0         0          0          0
#>   spec_prior         0      101         0        393          0
#>   very_recom         0        0         0          0          0
\end{Soutput}
\begin{Sinput}
accuracy <- sum(diag(confMat))/sum(confMat)
cat(sprintf("\nAccuracy=%f", accuracy))
\end{Sinput}
\begin{Soutput}
#> 
#> Accuracy=0.888717
\end{Soutput}
\end{Schunk}

\hypertarget{random-forest-model-fit}{%
\subsection{Random Forest model fit}\label{random-forest-model-fit}}

\begin{Schunk}
\begin{Sinput}
library(randomForest)
fitRF1 <- randomForest(as.factor(class)~.,
                      data=train.data, 
                      importance=TRUE, 
                      ntree=1000)
\end{Sinput}
\end{Schunk}

Importance of the dataset attributes for the prediction of the `class'
attribute shown in Figure \ref{fig:forimp}. Nam libero tempore, cum
soluta nobis est eligendi optio cumque nihil impedit quo minus id quod
maxime placeat facere possimus, omnis voluptas assumenda est, omnis
dolor repellendus. Temporibus autem quibusdam et aut officiis debitis
aut rerum necessitatibus saepe eveniet ut et voluptates repudiandae sint
et molestiae non recusandae. Itaque earum rerum hic tenetur a sapiente
delectus, ut aut reiciendis voluptatibus maiores alias consequatur aut
perferendis doloribus asperiores repellat.

\begin{Schunk}
\begin{Sinput}
varImpPlot(fitRF1, main="")
\end{Sinput}
\begin{figure}

{\centering \includegraphics{csda1010-lab1lab1_files/figure-latex/forimp-1} 

}

\caption[Importance of the dataset attributes for the prediction of the 'class' attribute]{Importance of the dataset attributes for the prediction of the 'class' attribute}\label{fig:forimp}
\end{figure}
\end{Schunk}

\hypertarget{random-forest-model-prediction-and-evaluation}{%
\subsection{Random Forest model prediction and
evaluation}\label{random-forest-model-prediction-and-evaluation}}

\begin{Schunk}
\begin{Sinput}
PredictionRF1 <- predict(fitRF1, test.data)
head(PredictionRF1)
\end{Sinput}
\begin{Soutput}
#>         7         9        33        48        53        60 
#>  priority not_recom not_recom not_recom  priority not_recom 
#> Levels: not_recom priority recommend spec_prior very_recom
\end{Soutput}
\end{Schunk}

\begin{Schunk}
\begin{Sinput}
confMat <- table(PredictionRF1,test.data$class)
confMat
\end{Sinput}
\begin{Soutput}
#>              
#> PredictionRF1 not_recom priority recommend spec_prior very_recom
#>    not_recom        432        0         0          0          0
#>    priority           0      425         0          0          0
#>    recommend          0        0         0          0          0
#>    spec_prior         0        1         0        404          0
#>    very_recom         0        0         0          0         32
\end{Soutput}
\begin{Sinput}
accuracy <- sum(diag(confMat))/sum(confMat)
cat(sprintf("\nAccuracy=%f", accuracy))
\end{Sinput}
\begin{Soutput}
#> 
#> Accuracy=0.999227
\end{Soutput}
\end{Schunk}

\hypertarget{dicsussion}{%
\section{Dicsussion}\label{dicsussion}}

Nor again is there anyone who loves or pursues or desires to obtain pain
of itself, because it is pain, but because occasionally circumstances
occur in which toil and pain can procure him some great pleasure. To
take a trivial example, which of us ever undertakes laborious physical
exercise, except to obtain some advantage from it? But who has any right
to find fault with a man who chooses to enjoy a pleasure that has no
annoying consequences, or one who avoids a pain that produces no
resultant pleasure?"

\hypertarget{conclusion}{%
\section{Conclusion}\label{conclusion}}

Lorem ipsum dolor sit amet, consectetur adipiscing elit, sed do eiusmod
tempor incididunt ut labore et dolore magna aliqua. Ut enim ad minim
veniam, quis nostrud exercitation ullamco laboris nisi ut aliquip ex ea
commodo consequat. Duis aute irure dolor in reprehenderit in voluptate
velit esse cillum dolore eu fugiat nulla pariatur. Excepteur sint
occaecat cupidatat non proident, sunt in culpa qui officia deserunt
mollit anim id est laborum.

This file is only a basic article template. For full details of
\emph{The R Journal} style and information on how to prepare your
article for submission, see the
\href{https://journal.r-project.org/share/author-guide.pdf}{Instructions
for Authors}.

\bibliography{RJreferences}


\address{%
Viviane Adohouannon\\
York University School of Continious Studies\\
\\
}
\url{https://learn.continue.yorku.ca/user/view.php?id=21444}

\address{%
Kate Alexander\\
York University School of Continious Studies\\
\\
}
\url{https://learn.continue.yorku.ca/user/view.php?id=21524}

\address{%
Juan Arangote\\
York University School of Continious Studies\\
\\
}
\url{https://learn.continue.yorku.ca/user/view.php?id=21472}

\address{%
Dian Azbel\\
York University School of Continious Studies\\
\\
}
\url{https://learn.continue.yorku.ca/user/view.php?id=20687}

\address{%
Igor Baranov\\
York University School of Continious Studies\\
\\
}
\url{https://learn.continue.yorku.ca/user/profile.php?id=21219}

