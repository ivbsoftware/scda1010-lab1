% !TeX root = RJwrapper.tex
\title{Ranking Applications for Nursery Schools Classification Problem}
\author{by Viviane Adohouannon, Kate Alexander, Diana Azbel, Igor Baranov}

\maketitle


% Any extra LaTeX you need in the preamble

\hypertarget{introduction}{%
\section{Introduction}\label{introduction}}

Malala Yousafzai once said, Let us remember: One book, one pen, one
child, and one teacher can change the world. Education is important
because it gives people the skills of basic literacy and numeracy, as
well as the ability to communicate, complete tasks and work with others.

Early schooling such as Nursery school education matters most as it
impacts children's long-term development and academic progress. While
all children benefit from a high-quality nursery school, family
structure, social and financial standing, and proximity to schools may
affect the enrollment process.

Nursery dataset \citep{noauthor_uci_nodate} was derived from a
hierarchical decision model originally developed to rank applications
for nursery schools. It was used during several years in 1980's when
there was excessive enrollment to these schools in Ljubljana, Slovenia,
and the rejected applications frequently needed an objective
explanation.

\hypertarget{background}{%
\subsection{Background}\label{background}}

Ljubljana is the capital and largest city of Slovenia. The city with an
area of 163.8 square kilometers is situated in the Ljubljana Basin in
Central Slovenia, between the Alps and the Karst. Ljubljana is located
some 320 kilometers south of Munich, 477 kilometers east of Zurich. In
1981 the population of the city rose to 224,817 inhabitants with
approximately 91\% of the population speaking Slovene as their primary
native language. The second most-spoken language is Bosnian, with
Serbo-Croatian being the third most-spoken language
\citep{noauthor_ljubljana_2018}.

During this time according to \citep{olave1989application} ``new housing
developments populated by young families, the demand for childrens
admission in nursery schools outstrips supply, notwithstanding the fast
growth rate of new schools.'' In this research, conducted in 1989, an
application of expert systems for admission procedures in public school
systems was presented. The specific problem under consideration was
selection of applicants for public nursery schools. The selection was
supported by an expert system which evaluates, classifies and ranks
applications. The main emphasis was on explanation of the underlying
knowledge and solutions, suggested by the system. The system has been
developed using DECMAK, an expert system shell for multi-attribute
decision making.

In continuation of the search for the practical solution of the problem,
second most relevant research was conducted in 1997
\citep{zupan1997machine}. It presented another machine learning method
that, given a set of training examples, induced a definition of the
target concept in terms of a hierarchy of intermediate concepts and
their definitions. This effectively decomposed the problem into smaller,
less complex problems. The method was inspired by the Boolean function
decomposition approach to the design of digital circuits. To cope with
high time complexity of inding an optimal decomposition, the authors
proposed a suboptimal heuristic algorithm.

It worth to notice that in both cases the researches concentrated on
designing and theoretical evaluation of mentioned algorithms without
proposing a practical solution that could be deployed in the
municipalities and put in use. The reason of this was a relatively high
at the time price of computers that could make predictictions with the
models proposed.

\hypertarget{objective}{%
\subsection{Objective}\label{objective}}

The objective of this article is to provide a reliable and feasible
recommendation algorithm to predict if a specific applicant is a
suitable candidate to be admitted into a nursery school or if the
applicant should stay with their parents. The results of this
recommendation may affect the childs engagement within the school,
parents involvement in school activities and overall satisfaction of the
applicants long-term academic progress.

\hypertarget{plan}{%
\subsection{Plan}\label{plan}}

To solve the objective a group of four students calling themselves The
First Group (T.F.G) from York University School of Continuing Studies,
have come together to create an algorithm using the Nursery dataset.
Since the target of this problem is a categorical value representing
recommendation if a child is suitable for the admittance to a nursery
school, a supervised classification algorithm was chosen
\citep{witten_data_2011}.

The main tool used in developing the recomendation algorithm is R
\citep{R}. The R language is widely used among statisticians and data
miners for developing statistical software and data analysis.

\hypertarget{data-understanding}{%
\section{Data understanding}\label{data-understanding}}

The dataset \citep{noauthor_uci_nodate} has 8 attributes and 12960
instances. Creators of the dataset suggested hierarchical model ranks
nursery-school applications according to the following concept
structure:

\begin{verbatim}
NURSERY Evaluation of applications for nursery schools 
. EMPLOY Employment of parents and child's nursery 
. . parents Parents' occupation 
. . has_nurs Child's nursery 
. STRUCT_FINAN Family structure and financial standings 
. . STRUCTURE Family structure 
. . . form Form of the family 
. . . children Number of children 
. . housing Housing conditions 
. . finance Financial standing of the family 
. SOC_HEALTH Social and health picture of the family 
. . social Social conditions 
. . health Health conditions 
\end{verbatim}

Nursery Database contains examples with the structural information
removed, i.e., directly relates NURSERY to the eight input attributes:
parents, has\_nurs, form, children, housing, finance, social, health.
Data set attribes presented in the following form:

\begin{verbatim}
parents: usual, pretentious, great_pret 
has_nurs: proper, less_proper, improper, critical, very_crit 
form: complete, completed, incomplete, foster 
children: 1, 2, 3, more 
housing: convenient, less_conv, critical 
finance: convenient, inconv 
social: non-prob, slightly_prob, problematic 
health: recommended, priority, not_recom
\end{verbatim}

Target attribute called ``class'' is a categorical variable having
several values that were not revealed in original dataset description
and had to be extracted from the data.

\hypertarget{preparation}{%
\subsection{Preparation}\label{preparation}}

To perform the analysis, certain R libraries were used. The code below
was used to load and initialize the libraries. The first line invoking
seed function was applied to enforce the repeatability of the
calculation results.

\begin{Schunk}
\begin{Sinput}
set.seed(42)
library(ggplot2)
library(rpart)
library(rpart.plot)
library(rattle)
library(caret)
\end{Sinput}
\end{Schunk}

The dataset was loaded directly from the dataset site
\citep{noauthor_uci_nodate} using the R statement below. Note that
column names were assigned as the online data did not have the header.
To pretty-print the head of the dataset xtable \citep{R-xtable} library
was used to generate Table \ref{table:dhead10}.

\begin{Schunk}
\begin{Sinput}
library(readr)
nursery_data <- 
  read.csv("http://archive.ics.uci.edu/ml/machine-learning-databases/nursery/nursery.data", 
  header = FALSE, 
  col.names = c("parents","has_nurs","form", "children","housing","finance",
                "social","health","class"))
\end{Sinput}
\end{Schunk}

\begin{table}[ht]
\centering
\scalebox{0.75}{
\begin{tabular}{rlllllllll}
  \hline
 & parents & has\_nurs & form & children & housing & finance & social & health & class \\ 
  \hline
1 & usual & proper & complete & 1 & convenient & convenient & nonprob & recommended & recommend \\ 
  2 & usual & proper & complete & 1 & convenient & convenient & nonprob & priority & priority \\ 
  3 & usual & proper & complete & 1 & convenient & convenient & nonprob & not\_recom & not\_recom \\ 
  4 & usual & proper & complete & 1 & convenient & convenient & slightly\_prob & recommended & recommend \\ 
  5 & usual & proper & complete & 1 & convenient & convenient & slightly\_prob & priority & priority \\ 
  6 & usual & proper & complete & 1 & convenient & convenient & slightly\_prob & not\_recom & not\_recom \\ 
  7 & usual & proper & complete & 1 & convenient & convenient & problematic & recommended & priority \\ 
  8 & usual & proper & complete & 1 & convenient & convenient & problematic & priority & priority \\ 
  9 & usual & proper & complete & 1 & convenient & convenient & problematic & not\_recom & not\_recom \\ 
  10 & usual & proper & complete & 1 & convenient & inconv & nonprob & recommended & very\_recom \\ 
   \hline
\end{tabular}
}
\caption{\tt Nursery Data Dataset (head)} 
\label{table:dhead10}
\end{table}

Summary of Nursery Data set is extracted by the R summary function, the
results are presented below.

\begin{Schunk}
\begin{Sinput}
summary(nursery_data)
\end{Sinput}
\begin{Soutput}
#>         parents            has_nurs            form      children   
#>  great_pret :4320   critical   :2592   complete  :3240   1   :3240  
#>  pretentious:4320   improper   :2592   completed :3240   2   :3240  
#>  usual      :4320   less_proper:2592   foster    :3240   3   :3240  
#>                     proper     :2592   incomplete:3240   more:3240  
#>                     very_crit  :2592                                
#>        housing           finance               social    
#>  convenient:4320   convenient:6480   nonprob      :4320  
#>  critical  :4320   inconv    :6480   problematic  :4320  
#>  less_conv :4320                     slightly_prob:4320  
#>                                                          
#>                                                          
#>          health            class     
#>  not_recom  :4320   not_recom :4320  
#>  priority   :4320   priority  :4266  
#>  recommended:4320   recommend :   2  
#>                     spec_prior:4044  
#>                     very_recom: 328
\end{Soutput}
\end{Schunk}

Taking a quick look at the summary it is noticed that target attribute
`class' is presented in the dataset by five values that appear logically
in the order of recommendation strength:

\begin{itemize}
\tightlist
\item
  not\_recom
\item
  recommend
\item
  very\_recom
\item
  priority
\item
  special\_prior
\end{itemize}

Note that `recommend' value is presented in the dataset by only two
rows. Let's look at the distribution of values. The frequency of the
segment ``recommend'' is low (0.00015\%) and will not be have effect on
the results of prediction algorithms.

\begin{Schunk}
\begin{Sinput}
prop.table((table(nursery_data$class)))
\end{Sinput}
\begin{Soutput}
#> 
#>   not_recom    priority   recommend  spec_prior  very_recom 
#> 0.333333333 0.329166667 0.000154321 0.312037037 0.025308642
\end{Soutput}
\end{Schunk}

The following code that uses graphical R ggplot library
\citep{R-ggplot2} gererates a graphical presentation of this
distribution.

\begin{Schunk}
\begin{Sinput}
ggplot(nursery_data, aes(x=class)) + 
  geom_bar(aes(y= (..count..)/sum(..count..)),color="blue",fill=rgb(0.2,0,0.5)) +
  theme(legend.position = "none") +
  scale_y_continuous(labels=scales:::percent) +
  labs(x = "class",y="total")
\end{Sinput}
\begin{figure}[h]

{\centering \includegraphics{csda1010-lab1lab1_files/figure-latex/classd-1} 

}

\caption[Distribution of Class attribute]{Distribution of Class attribute}\label{fig:classd}
\end{figure}
\end{Schunk}

Figure \ref{fig:classd} shows the targeted attribute with five segments.
From the data snapshot, 33\% of nursery school applications ended not
being recommended. As expected, `recommend' value has almost no visible
presentation. It is worth noticing that presentation of `very\_recom'
value is also very low in the dataset and almost ten times lower than
other values. This might be an issue that could affect the accuracy of
the model and most likely will have to be taken care of later. We can
also conclude that there is no missing information and dataset is in
good standing to be used.

The following code rendered to (Figure \ref{fig:classdp}) shows
distribution of class attribute depending on parents attribute. From
this graph, it clearly shows that `priority' and `recommendation' are
given to applications from `pretentious' parents while recommending kids
with usual parents to stay home.

\begin{Schunk}
\begin{Sinput}
ggplot(nursery_data,aes(x=class, fill=parents))+
  geom_bar(position="dodge")
\end{Sinput}
\begin{figure}[h]

{\centering \includegraphics{csda1010-lab1lab1_files/figure-latex/classdp-1} 

}

\caption[Distribution of Class attribute depending on parents]{Distribution of Class attribute depending on parents}\label{fig:classdp}
\end{figure}
\end{Schunk}

The following code rendered to (Figure \ref{fig:classdh}) shows
distribution of class attribute depending on child `health' attribute.
Being healthy is most important and relevant for parents to have their
nursery application being recommended. Parents having strong and good
social situation are more likely to have their application recommended
than the ones who do not.

\begin{Schunk}
\begin{Sinput}
nursery_data$health <- as.factor(nursery_data$health)
ggplot(data = nursery_data, mapping = aes(x = class, fill = health)) +
  geom_bar()
\end{Sinput}
\begin{figure}[h]

{\centering \includegraphics{csda1010-lab1lab1_files/figure-latex/classdh-1} 

}

\caption[Distribution of Class attribute depending on health]{Distribution of Class attribute depending on health}\label{fig:classdh}
\end{figure}
\end{Schunk}

To analyze distribution of class attribute depending on `social' and
`parents' attributes the following code renders (Figure
\ref{fig:classdi}).

\begin{Schunk}
\begin{Sinput}
ggplot(nursery_data, aes(class, fill=social)) + 
  geom_bar(aes(y = (..count..)/sum(..count..)), alpha=0.9) +
  facet_wrap(~parents) + 
  scale_fill_brewer(palette = "Dark2", direction = -1) +
  scale_y_continuous(labels=scales:::percent, breaks=seq(0,0.4,0.05)) +
  ylab("Percentage") +
  theme_bw() +
  theme(plot.title = element_text(hjust = 0.5)) +
  theme(axis.text.x = element_text(angle = 90, hjust = 1))
\end{Sinput}
\begin{figure}[h]

{\centering \includegraphics{csda1010-lab1lab1_files/figure-latex/classdi-1} 

}

\caption[Distribution of Class attribute depending on income]{Distribution of Class attribute depending on income}\label{fig:classdi}
\end{figure}
\end{Schunk}

The following code (Figure \ref{fig:classdn}) renders distribution of
`class' attribute depending on `has\_nurs' and `form' attribute. Parents
in segments ``proper'' and ``less\_proper'' appear to have similar
distribution and their application taken as priority.

\begin{Schunk}
\begin{Sinput}
ggplot(nursery_data, aes(class, fill=form)) + 
  geom_bar(aes(y = (..count..)/sum(..count..)), alpha=0.9) +
  facet_wrap(~has_nurs) + 
  scale_fill_brewer(palette = "Dark2", direction = -1) +
  scale_y_continuous(labels=scales:::percent, breaks=seq(0,0.4,0.05)) +
  ylab("Percentage") +
  theme_bw() +
  theme(plot.title = element_text(hjust = 0.5)) +
  theme(axis.text.x = element_text(angle = 90, hjust = 1))
\end{Sinput}
\begin{figure}[h]

\includegraphics{csda1010-lab1lab1_files/figure-latex/classdn-1} \hfill{}

\caption[Distribution of Class attribute depending on nursery]{Distribution of Class attribute depending on nursery}\label{fig:classdn}
\end{figure}
\end{Schunk}

To analyze the interdependance of different attributes and their
influence on the target ``class'' attribute does not provide as a clear
picture of the importance of those attributes. The exclusion of the
`health' attribute seems to have a visible correlation with the target.
The more sophisticated method should be applied to evaluate the
attribute importance.

\hypertarget{modeling}{%
\section{Modeling}\label{modeling}}

\hypertarget{splitting-the-dataset-into-train-and-test}{%
\subsection{Splitting the dataset into train and
test}\label{splitting-the-dataset-into-train-and-test}}

The Nursery dataset has been split in such a way that train and test
sets would have the same distribution of the `class' attribute. The
reason for this stratification strategy is to focus on the priority of
an applicants placement in a nursery school rather than an applicants
family or social status. We used 75:25 split ratio.

\begin{Schunk}
\begin{Sinput}
train.rows<- createDataPartition(y= nursery_data$class, p=0.75, list = FALSE)
train.data<- nursery_data[train.rows,]
prop.table((table(train.data$class)))
\end{Sinput}
\begin{Soutput}
#> 
#>    not_recom     priority    recommend   spec_prior   very_recom 
#> 0.3332990433 0.3291842403 0.0002057402 0.3120049378 0.0253060385
\end{Soutput}
\end{Schunk}

\begin{Schunk}
\begin{Sinput}
test.data<- nursery_data[-train.rows,]
prop.table((table(test.data$class)))
\end{Sinput}
\begin{Soutput}
#> 
#>  not_recom   priority  recommend spec_prior very_recom 
#> 0.33343625 0.32911392 0.00000000 0.31213337 0.02531646
\end{Soutput}
\end{Schunk}

The code below (Figure \ref{fig:testd}) renders distribution of `class'
attribute depending on `nursery' attribute. Not suprisingly, the test
set did not get any of rows with ``recommend'' class attribute.

\begin{Schunk}
\begin{Sinput}
ggplot(test.data, aes(x=as.factor(class))) + 
  geom_bar(aes(y = (..count..)/sum(..count..)),width=0.4,
  color="red", fill=rgb(0.9,1,0.7) )+theme(legend.position = "none") + 
  labs(x = "class",y="total")+scale_y_continuous(labels=scales:::percent)
\end{Sinput}
\begin{figure}[h]

{\centering \includegraphics{csda1010-lab1lab1_files/figure-latex/testd-1} 

}

\caption[Distribution of Class attribute depending on nursery]{Distribution of Class attribute depending on nursery}\label{fig:testd}
\end{figure}
\end{Schunk}

\hypertarget{decision-tree-model-fit}{%
\subsection{Decision Tree model fit}\label{decision-tree-model-fit}}

As a first step we have used a Decision Tree model \citep{R-rpart}.
Though this model is known to be computationally fast but not very
precise, we have use all default parameters and all attributes of the
train dataset in the resulting Decision Tree presented in Figure
\ref{fig:dtree}.

\begin{Schunk}
\begin{Sinput}
fitdt <- rpart(
  as.factor(class)~parents+has_nurs+form+children+housing+finance+social+health,
  method="class", data=train.data)
fancyRpartPlot(fitdt, main="", sub="")
\end{Sinput}
\begin{figure}[h]

{\centering \includegraphics{csda1010-lab1lab1_files/figure-latex/dtree-1} 

}

\caption[Decision tree diagram]{Decision tree diagram}\label{fig:dtree}
\end{figure}
\end{Schunk}

\hypertarget{decision-tree-model-evaluation}{%
\subsection{Decision Tree model
evaluation}\label{decision-tree-model-evaluation}}

This Decision Tree favours the following attributes in order of their
importance for the prediction of the target attribute: health,
has\_nurs, parents. It does not consider the rest of the attributes as
important. Let's apply the model to the test set and evaluate accuracy
of the predictions.

\begin{Schunk}
\begin{Sinput}
dtPrediction <- predict(fitdt, test.data, type = "class")
\end{Sinput}
\end{Schunk}

\begin{Schunk}
\begin{Sinput}
confMat <- table(dtPrediction,test.data$class)
confMat
\end{Sinput}
\begin{Soutput}
#>             
#> dtPrediction not_recom priority recommend spec_prior very_recom
#>   not_recom       1080        0         0          0          0
#>   priority           0      836         0         96         82
#>   recommend          0        0         0          0          0
#>   spec_prior         0      230         0        915          0
#>   very_recom         0        0         0          0          0
\end{Soutput}
\begin{Sinput}
accuracy <- sum(diag(confMat))/sum(confMat)
cat(sprintf("\nAccuracy=%f", accuracy))
\end{Sinput}
\begin{Soutput}
#> 
#> Accuracy=0.874035
\end{Soutput}
\end{Schunk}

As we can see, the models predicted all ``not\_recom'' values correctly,
wrongly predicted all ``very\_recom'' values and had partial success
predicting ``priority'' and ``special\_priority'' values. The overall
accuracy is 87\%, but the model is not precise and not usable.

\hypertarget{random-forest-model-fit}{%
\subsection{Random Forest model fit}\label{random-forest-model-fit}}

Next step is to use more advanced but more computationally demanding
Random Forest method \citep{R-randomForest}:

\begin{Schunk}
\begin{Sinput}
library(randomForest)
fitRF1 <- randomForest(
  as.factor(class)~parents+has_nurs+form+children+housing+finance+social+health,
  data=train.data, importance=TRUE, ntree=1000)
\end{Sinput}
\end{Schunk}

\begin{Schunk}
\begin{Sinput}
varImpPlot(fitRF1, main="")
\end{Sinput}
\begin{figure}[h]

{\centering \includegraphics{csda1010-lab1lab1_files/figure-latex/forimp-1} 

}

\caption[Importance of the dataset attributes for the prediction of the 'class' attribute]{Importance of the dataset attributes for the prediction of the 'class' attribute}\label{fig:forimp}
\end{figure}
\end{Schunk}

Importance of the dataset attributes for the prediction of the ``class''
attribute shown in Figure \ref{fig:forimp}. Analyzing this chart we
conclude that the proper order of attributes importance for the
prediction of the target attribute is: health, has\_nurs, parents,
housing, social, children. The last two attributes are less important.
This contradicts the original analysis \citep{noauthor_uci_nodate} where
it is assumed that parents and finance are the most important attributes
and health is the least. The only way to solve this dilema is to
calculate accuracy of our model.

\newpage

\hypertarget{random-forest-model-prediction-and-evaluation}{%
\subsection{Random Forest model prediction and
evaluation}\label{random-forest-model-prediction-and-evaluation}}

\begin{Schunk}
\begin{Sinput}
PredictionRF1 <- predict(fitRF1, test.data)
\end{Sinput}
\end{Schunk}

\begin{Schunk}
\begin{Sinput}
confMat <- table(PredictionRF1,test.data$class)
confMat
\end{Sinput}
\begin{Soutput}
#>              
#> PredictionRF1 not_recom priority recommend spec_prior very_recom
#>    not_recom       1080        0         0          0          0
#>    priority           0     1046         0         14         37
#>    recommend          0        0         0          0          0
#>    spec_prior         0       20         0        997          0
#>    very_recom         0        0         0          0         45
\end{Soutput}
\begin{Sinput}
accuracy <- sum(diag(confMat))/sum(confMat)
cat(sprintf("\nAccuracy=%f", accuracy))
\end{Sinput}
\begin{Soutput}
#> 
#> Accuracy=0.978080
\end{Soutput}
\end{Schunk}

The models predicted ``not\_recom'' values correctly, ``priority'' and
``special\_priority'' values were predicted with about 90\% accuracy. As
expected, ``very\_recom'' value was predicted poorly - with almost 40\%
error. The overall accuracy is much better at 97.8\%, but the model is
still not precise and not usable.

\newpage

\hypertarget{imballanced-classification-correction}{%
\subsection{Imballanced Classification
Correction}\label{imballanced-classification-correction}}

To improve the precision of the model we need to take care of the
imballanced distribution of rows with target value ``very\_recom''.
There are several techniques to correct the problem of low accuracy in
regards to predicting ``very\_recom'' value of target attribute
``class'' \citep{noauthor_how_2017}. One particular method that we
decided to employ is method called Random Over-Sampling, which increases
the number of instances in the minority class by randomly replicating
them in order to present a higher representation of the minority class
in the sample. This method leads to no information loss, but increases
the likelihood of overfitting since it replicates the minority class
events. To decrease the latter, we will apply this technique only to
train data. As you can see from the output of the code below, the
portion of ``very\_recom'' value of the ``class'' attribute in new
train2.data dataset is now about 30\% comparing to previous 7\% of the
rest of the values (excluding ``recommend'' of course).

\begin{Schunk}
\begin{Sinput}
vr <- train.data[train.data[,"class"] == "very_recom", ]
train2.data <- train.data
for (i in 1:3) {
  train2.data <- rbind(train2.data, vr)  
}
prop.table((table(train2.data$class)))
\end{Sinput}
\begin{Soutput}
#> 
#>    not_recom     priority    recommend   spec_prior   very_recom 
#> 0.3097810498 0.3059565924 0.0001912229 0.2899894827 0.0940816522
\end{Soutput}
\end{Schunk}

\hypertarget{corrected-random-forest-model-prediction-and-evaluation}{%
\subsection{Corrected Random Forest model prediction and
evaluation}\label{corrected-random-forest-model-prediction-and-evaluation}}

\begin{Schunk}
\begin{Sinput}
library(randomForest)
fitRF2 <- randomForest(
  as.factor(class)~parents+has_nurs+form+children+housing+finance+social+health,
  data=train2.data, importance=TRUE, ntree=1000)
\end{Sinput}
\end{Schunk}

Now let's calculate prediction and it's evaluation using the corrected
dataset. As we expected, now all but one of the ``very\_recom'' values
of the target ``class'' attribute were predicted correctly which led to
total accuracy of the prediction to 99\% with overall balanced precision
more that 97\% accross all the predicted ``class'' values.

\begin{Schunk}
\begin{Sinput}
PredictionRF2 <- predict(fitRF2, test.data)
confMat2 <- table(PredictionRF2,test.data$class)
confMat2
\end{Sinput}
\begin{Soutput}
#>              
#> PredictionRF2 not_recom priority recommend spec_prior very_recom
#>    not_recom       1080        0         0          0          0
#>    priority           0     1036         0         15          1
#>    recommend          0        0         0          0          0
#>    spec_prior         0       28         0        996          0
#>    very_recom         0        2         0          0         81
\end{Soutput}
\begin{Sinput}
accuracy <- sum(diag(confMat2))/sum(confMat2)
cat(sprintf("\nAccuracy=%f", accuracy))
\end{Sinput}
\begin{Soutput}
#> 
#> Accuracy=0.985798
\end{Soutput}
\end{Schunk}

\newpage

\hypertarget{conclusion}{%
\section{Conclusion}\label{conclusion}}

Through exploring the Nursery dataset and using a classification model
to developed our algorithm to predict applicants suitabilty of an
admittance within the nursery school system, the evaluation of the model
was developed using Random Forest algorithm. Even though the original
dataset was clear and did not have missing values, it was unballanced.
To overcome this a method called Random Over-Sampling was appplied,
which increased the number of instances in the minority class by
randomly replicating them in order to present a higher representation of
the minority class in the sample. The final model achieved almost 99\%
accuracy of predictions with overall balanced precision more that 97\%
accross all the predicted ``class'' values. The project was a success.

\bibliography{RJreferences}

\newpage

\hypertarget{note-from-the-authors}{%
\section{Note from the Authors}\label{note-from-the-authors}}

This file was generated using
\href{https://github.com/rstudio/rticles}{\emph{The R Journal} style
article template}, additional information on how to prepare articles for
submission is here -
\href{https://journal.r-project.org/share/author-guide.pdf}{Instructions
for Authors}. The article itself is an executable R Markdown file that
could be
\href{https://github.com/ivbsoftware/scda1010-lab1/tree/master/docs/R_Journal/csda1010-lab1lab1}{downloaded
from Github} with all the necessary artifacts.


\address{%
Viviane Adohouannon\\
York University School of Continuing Studies\\
\\
}
\url{https://learn.continue.yorku.ca/user/view.php?id=21444}

\address{%
Kate Alexander\\
York University School of Continuing Studies\\
\\
}
\url{https://learn.continue.yorku.ca/user/view.php?id=21524}

\address{%
Diana Azbel\\
York University School of Continuing Studies\\
\\
}
\url{https://learn.continue.yorku.ca/user/view.php?id=20687}

\address{%
Igor Baranov\\
York University School of Continuing Studies\\
\\
}
\url{https://learn.continue.yorku.ca/user/profile.php?id=21219}

